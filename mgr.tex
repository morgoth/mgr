\documentclass[a4paper,12pt]{article}

\usepackage[utf8]{inputenc}
\usepackage{polski}
\usepackage[polish]{babel}
\usepackage{graphicx}
\usepackage{url}
\usepackage{amsfonts}
\usepackage{amsmath}
\usepackage{float}

\input{pygments}

\providecommand{\imref}[1]{Rys. \ref{#1}} % referencja do obrazka

%% Define a new 'leo' style for the package that will use a smaller font.
\makeatletter
\def\url@leostyle{
  \@ifundefined{selectfont}{\def\UrlFont{\sf}}{\def\UrlFont{\small\ttfamily}}}
\makeatother
%% Now actually use the newly defined style.
\urlstyle{leo}

\newfloat{listing}{thp}{lol}
\floatname{listing}{Listing}

\begin{document}
\author{Wojciech Wnętrzak}
\title{Opracowanie elastycznego systemu organizacji testów internetowych do
zdalnej edukacji w oparciu o technologię Ruby on Rails}
\date{\today}

\begin{titlepage}
\maketitle
\end{titlepage}

\section{Pierwszy rozdział}

\end{document}
