\documentclass[a4paper,12pt]{article}

\usepackage[utf8]{inputenc}
\usepackage{polski}
\usepackage[polish]{babel}
\usepackage{graphicx}
\usepackage{url}
\usepackage{amsfonts}
\usepackage{amsmath}
\usepackage{float}

\input{pygments}

\providecommand{\imref}[1]{Rys. \ref{#1}} % referencja do obrazka

%% Define a new 'leo' style for the package that will use a smaller font.
\makeatletter
\def\url@leostyle{
  \@ifundefined{selectfont}{\def\UrlFont{\sf}}{\def\UrlFont{\small\ttfamily}}}
\makeatother
%% Now actually use the newly defined style.
\urlstyle{leo}

\newfloat{listing}{thp}{lol}
\floatname{listing}{Listing}

\begin{document}
\author{Wojciech Wnętrzak}
\title{Opracowanie elastycznego systemu organizacji testów internetowych do
zdalnej edukacji w oparciu o technologię Ruby on Rails}
\date{\today}

\begin{titlepage}
\maketitle
\end{titlepage}

\section{Odległość Levenshteina}
Jednym z rodzajów tworzonego pytania jest pole tekstowe. Po wyborze tego właśnie typu, w
treści odpowiedzi wpisujemy wzorzec, który będzie porównany z odpowiedzią wpisaną przez
studenta.\\
Porównanie odbywa się na zasadzie przyrównywania kolejnych liter wzorca do otrzymanego
rozwiązania. Pojawia się jednak problem z tzw. „literówkami”. Co w przypadku gdy student
napisze dane słowo z dużej litery lub pomyli kolejność liter (często zdarzający się błąd)?
W celu rozwiązania powyższej przeszkody, przyjąłem dwa uproszczenia, mianowicie przy
porównaniu:
\begin{itemize}
  \item nie będzie brana pod uwagę wielkość liter
  \item na każde 6 liter wzorca jest możliwy jeden błąd
\end{itemize}
Do realizacji drugiego podpunktu została użyta odległość Levenshteina, która docelowo
oblicza wartość liczbową pomiędzy dwoma ciągami znaków, dzięki czemu jesteśmy w stanie
ocenić jak bardzo różnią się od siebie porównywane napisy.\\
Definicję można przedstawić w następujący sposób:\\
Odległością pomiędzy dwoma napisami jest najmniejsza liczba działań prostych,
przeprowadzających jeden napis w drugi, gdzie działaniem prostym nazwiemy:
\begin{itemize}
  \item wstawienie nowego znaku do napisu
  \item usunięcie znaku z napisu
  \item zamianę znaku w napisie na inny znak
\end{itemize}
Mając obliczoną odległość pomiędzy wzrocem a odpowiedzią, mierzymy długość wzorca i
dzielimy przez 6, jeśli wynik jest większy bądź równy od odległości, uznajemy odpowiedź
za poprawną. Liczba 6 została zaczerpnięta z polskiej wersji portalu NerdQuiz, gdzie
zastosowany jest podobny mechanizm oceny.
\end{document}
